
\addcontentsline{toc}{section}{\textcolor{cyan}{Conclusion générale}}
\begin{flushleft}
	\section*{\textcolor{cyan}{Conclusion générale}}
	Dans ce projet nous nous sommes concentrés à la commande et la surveillance d'un système à distance utilisant un protocole de communication sans fil (WiFi) avec un Smartphone.\newline
	
	Au début on a essayé de connecter le système au réseau internet on a utilisé la connexion sans fil (par WiFi) à travers la carte NodeMCU. Pour la commande et la visualisation des grandeurs 
	physiques mesurées on a exécuté un programme capable de commander une pompe et de 
	visualiser des grandeurs physique tel que la température, humidité, humidité de sol, lumière et 
	niveau d’eau d’un réservoir sur l’application de Smart-phone.\newline
	
	En deuxième lieu, nous avons développé une interface de commande et de supervision sous Androide (une application sur Smart-phone) avec la framework flutter. Nous 
	avons préparé les icônes, les labelles et les graphe pour permettre au utilisateur de superviser 
	les résultats attendus programmés dans le système embarqué.\newline
	
	Une telle réalisation n’est pas dénuée de difficultés. Il est à noter que nous nous sommes 
	confrontés à plusieurs problèmes surtout dans la partie de la connexion sans fil, 
	l’implémentation des représentation graphique et l’exécution en temps réel dont ce problème 
	est tout à fait lié au serveur Firebase. Cependant, on peut dire que malgré ces difficultés, les 
	résultats obtenus à travers cette étude qu’ils soient pratiques ou théoriques, permettent d’ouvrir 
	la porte à d’autres études. Nous espérons que ce rapport sera une référence aux personnes 
	désirant développer et réaliser des projets et systèmes à base d’une carte NodeMCU.\newline
	
	Comme perspectives, nous proposons l’amélioration de ce travail par les actions suivantes :\newline
	
	\begin{itemize}
		\item Envoyer une notification lorsque la valeur d’humidité de sol atteint une certaine valeur.
	\end{itemize}
	\begin{itemize}
		\item Ajouter d’autre option dans la visualisation graphique des données comme voir les 
		\item 	valeurs par jour, semaine, heure…
	\end{itemize}
	\begin{itemize}
		\item Gérer plus qu’un utilisateur.
	\end{itemize}
	
\end{flushleft}

\newpage
