
\addcontentsline{toc}{section}{\textcolor{cyan}{\textbf{Introduction générale}}}
\begin{flushleft}
\section*{\textcolor{cyan}{Introduction générale}}
 Avec la croissance exponentielle de la population mondiale,selon l'Organisation des Nations Unies pour l'alimentation et l'agriculture, le monde devra produire 70\% de nourriture après 2050, la diminution des terres agricoles et l'épuisement des ressources naturelles limitées, la nécessité d'améliorer le rendement agricole est devenue critique. De plus, la main-d'œuvre agricole dans la plupart des pays a diminué. Donc l'adoption de solutions de connectivité Internet dans les pratiques agricoles a été déclenchée, afin de réduire le besoin de travail manuel.
 
 L'agriculture intelligente basée sur les technologies IoT permet aux agriculteurs de réduire les déchets et d'améliorer la productivité allant de la quantité d'engrais utilisée au une utilisation efficace des ressources telles que l'eau,  	 	 l'électricité, etc. La solution agriculture intelligente est 
 un système conçu pour surveiller le champ de culture à l'aide des capteurs (lumière, humidité,température, humidité du sol, etc.) et automatiser le système d'irrigation.
  
 L’objectif de notre projet est la réalisation d'un prototype d'un système irrigation connecter pour une application 
 et la commande à distance de processus d’irrigation des plantes. A l’aide des capteurs, l’application permet à l’utilisateur de superviser 3 paramètres météorologique (Température,Humidité relative, Lumière) et détecter le besoin d’eau de la plante et aussi la surveillance de niveau d’eau d’un réservoir. 
 
 Ce rapport présente l’ensemble des étapes suivies pour réaliser ce projet. Il contient 4 parties organisés comme suit :
 Le premier partie présente le contexte général et le cahier de charge de notre projet. Le deuxième partie présente les différentes technologies utilisées pour réaliser le prototype. Le troisième partie présente des mini-projets réaliser afin de se familiariser avec les technologies citées dans le chapitre précédente. Le quatrième partie présente la solution 
 Smart Agri. Enfin une conclusion générale.
\end{flushleft}
\bibliographystyle{plain}
\newpage
	